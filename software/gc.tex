\documentclass[11pt]{article}
\usepackage{hyperref}
\begin{document}
\pagestyle{myheadings}
\markboth{GENECOUNTING documentation}{GENECOUNTING documentation}
\thispagestyle{empty}

{\noindent\bf GENECOUNTING: Haplotype Analysis with Missing
Genotypes}

\medskip\noindent \copyright Copyright 2001-8 Jing Hua Zhao

\bigskip\noindent GENECOUNTING implements an EM algorithm for
haplotype analysis of unrelated subjects (Zhao et al.  2002).  It
has been recently extended to handle data on X chromosome so both
males and females can be used in a single analysis. The algorithm
can handle individuals with missing genotype data.  However,
currently it is limited to missing with both alleles at these loci.

\medskip\noindent This distribution also contains a module
GENECOUNTING/PREPARE for preparing input data to GENECOUNTING and a
module GENECOUNTING/PERMUTE obtaining permutation tests for global
association and significance of specific haplotypes using
Freeman-Tukey and proportion tests.

\section{Installation}

\medskip\noindent GENECOUNTING program and associates can be obtained
from the following URLs:

\medskip
\url{http://www.mrc-epid.cam.ac.uk/~jinghua/software.htm}
\medskip

\medskip\noindent They are distributed in a zip file, so you will
need unzip them first. e.g. by unzip

\medskip
{\bf unzip gc22.zip}
\medskip

\medskip\noindent to a directory under both Unix and Windows.  You
can also use WINZIP (available from \url{http://www.winzip.com}).
Full description of all files contained is available in packing.lst.


\section{Input}

\medskip\noindent Assuming there are n markers with alleles a1, a2, ..., an,
the program accepts raw genotype data in the following format,

\begin{verbatim}
line 1: a1 a2 ... an
line 2- ID w <marker 1 genotype> ... <marker n genotype>
\end{verbatim}
\medskip\noindent where w is a column of weights associated with specific
multilocus genotypes. Marker genotype at each marker consists of two
integers indicating numbered alleles.  Nonpositive integers indicate
missing genotypes.

\medskip\noindent For data involves X chromosome, an extra column
is required to indicate sex (0=male, 1=female) of each individual
(record), so that the format above becomes,

\begin{verbatim}
line 1: a1 a2 ... an
line 2- ID w <sex> <marker 1 genotype> ... <marker n genotype>
\end{verbatim}

\medskip\noindent Note that marker genotype could either be ``allele 1
allele 2'' or ``allele 1/allele 2'' if two alleles are involved at a
locus.


\section{Output}

\medskip\noindent GENECOUNTING reports number of individuals with
genotypes at each marker, allele frequencies, haplotype frequency
estimates assuming equilibrium and disequilibrium and their
associated log-likelihoods. It also gives haplotype assignment and
posterior probabilities for all subjects based on their observed
multilocus genotypes.


\section{Running GENECOUNTING}

\medskip\noindent The command syntax is as follows,

\medskip
{\bf gc $<$input file$>$ [output file] [threshold]}

\medskip
\medskip\noindent where $<$input file$>$ contains the raw genotypes and
[output file] contains result of the analysis. [threshold] is a
cutoff value for trimming posterior probabilities, with default
value 0.001. Both output file and threshold are optional so they are
put in []. Without specifying output file the output will be sent to
computer screen, and can be redirected to a text file. e.g.

\medskip
{\bf gc my-input-file $>$ my-screen-output}
\medskip

\medskip\noindent The format with X chromosome version for gcx is similar.


\section{Examples}

\medskip\noindent A simulated data of four SNPs is contained in file
4snps.4m, whose first few lines are shown below,

\begin{verbatim}
 2  2  2  2
    1    29  1  1  1  1  1  1  1  1
    2     9  1  1  1  1  1  1  1  2
    3     2  1  1  1  1  1  1  2  2
    4     4  1  1  1  1  1  1  0  0
    5    67  1  1  1  1  1  2  1  1
    ...
\end{verbatim}

\medskip\noindent where Line 1 contains 4 integers representing numbers of
alleles at four SNPs Line 2- contain the following columns: Column
1, sequence of numbers associated with 78 observed multilocus
genotypes; Column 2, number of subjects with these genotypes;
Columns 3-10, the actual genotypes.

\medskip\noindent For example, line 2 shows there are 29 individuals
homozygous at all four loci, line 3 shows there are 9 individuals
homozygous at the first three SNPs.

\medskip\noindent The following command can be used to obtain haplotype
frequency estimates:

\medskip
{\bf gc 4snps.4m 4snps.out}
\medskip

\medskip\noindent The HLA data as described in Zhao et al.  (2000), Zhao and
Sham (2002) is provided as file hla.dat.  The example is to
illustrate haplotype reconstruction with both missing data and
multiallelic markers.

\medskip\noindent An example file involving X chromosome data is mao.inp.


\section{Utility programs}

\subsection{GENECOUNTING/PREPARE}

\medskip\noindent This program is able to condense the raw genotype data
into the form as required by GENECOUNTING and has the following
command syntax,

\medskip
{\bf pgc $<$parameter file$>$ $<$data file$>$ $<$output file$>$}
\medskip

\medskip\noindent where <parameter file> and <data file> are EHPLUS (Zhao et
al. 2000, also available from the URL above) parameter and data
files and collapsing individuals with similar information into
single category and recording number of instances occurs.  Always
select option for marker-marker analysis to prepare input file for
GENECOUNTING.

\medskip\noindent Briefly, the data file can be in the form of either
alleles

\medskip
       {\bf [ID] [label]  [1a]   [1b]   [2a]   [2b] ...}
\medskip

\medskip\noindent or genotype identifiers

\medskip
       {\bf [ID] [label]  [1]    [2] ...}
\medskip

\medskip\noindent where [ID] and [label] are individual's ID and
case-control status, and [1a], [1b], [2a], [2b] are pairs of
numbered alleles at each marker separated by spaces.  As a genotype
identifier of a marker uniquely determines both alleles it is also
accepted.  For instance a SNP the genotype identifiers for marker
genotypes 1/1, 1/2 and 2/2 are 1, 2, and 3, respectively.  In
general, let $L$ and $U$ $(L<=U)$ be the two alleles at a marker,
then the identifier is calculated as follows $$ L+U(U-1)/2$$

\medskip\noindent File 4snps.dat contains 207 cases and 225 controls
genotyped at four SNPs and can be used to generate 4snps.4m.  Some
lines of the file are shown as follows.

\begin{verbatim}
  1 1      1 2     1 2     2 2     1 2
  2 1      2 2     2 2     2 2     2 2
  3 1      1 2     0 0     1 2     1 2
  4 1      1 1     1 1     1 1     1 1
  5 1      1 1     1 1     2 2     1 2
...

208 0      1 1     1 1     1 2     1 1
209 0      1 1     1 1     0 0     1 1
210 0      1 1     1 1     1 2     1 2
211 0      1 2     1 1     1 1     1 1
212 0      0 0     1 1     1 1     1 2
...
\end{verbatim}

\medskip\noindent where

\medskip
\medskip\noindent Column 1 is individual's ID

\medskip\noindent Column 2 is a label showing the individual to be case (=1)
or control (=0)

\medskip\noindent Columns 3-10 are the actual genotypes
\medskip

\medskip\noindent It is simply the raw genotype data plus columns 1 and 2
showing individual IDs and case/control labels.

\medskip\noindent The parameter file consists of six lines indicating basic
information of the data file and the analysis to be performed. This
makes it possible to generate data for analysis involving only
subset of markers.  A parameter file appropriate for file 4snps.dat
is 4snps.par containing the following lines,

{\small
\begin{verbatim}
4 0 0 0     << number of loci, case/control label, label permutation, replicates
2 2 2 2     << number of alleles
0 0         << raw genotype data, suppress screen reports
1 1 1 1     << select all markers for analysis
0 0 0 0     << no permutation for all 4 SNPs
0.001 0.05 0.2 0.8 << a putative disease model
\end{verbatim}
}

\medskip\noindent Use the following command

\medskip
{\bf pgc 4snps.par 4snps.dat 4snps.4m}
\medskip

\noindent to obtain 4snps.4m

\medskip\noindent It is possible to make the program to use raw genotype
directly, as shown by 4snp.inp, whose first few lines of are shown
as follows,

\begin{verbatim}
2 2 2 2
  1  1   1 2     1 2     2 2     1 2
  2  1   2 2     2 2     2 2     2 2
  3  1   1 2     0 0     1 2     1 2
  4  1   1 1     1 1     1 1     1 1
  5  1   1 1     1 1     2 2     1 2
  ...
\end{verbatim}

\medskip\noindent Note all individuals have weight 1 at column 2.  Now the
command to perform the analysis becomes,

\medskip
{\bf gc 4snps.inp 4snps.out}
\medskip

\medskip\noindent While this is rather straightforward, it is more
time-consuming for running the analysis.  It is also important that
individuals with no information are excluded from the data file.


\subsection{GENECOUNTING/PERMUTE}

\medskip\noindent This module has been available from version 1.3. It has
the following command syntax,

\medskip
\noindent{\small\bf gcp $<$parameter file$>$ $<$data file$>$
[$<$output file$>$ [random number} seed]]
\medskip

\medskip\noindent where $<$parameter file$>$ and $<$data file$>$ are EHPLUS
format files as before, while the optional $<$output file$>$
specifies the name of the output file. If $<$output file$>$ is
omitted, then the output will be sent to the computer screen. The
seed for pseudorandom number generator can be specified after
$<$output file$>$; its default value is 3,000.

\medskip\noindent For case-control data, it generates a sequence of
heterogeneity statistics by permuting case-control labels. For a
marker-marker analysis, it tests for linkage disequilibrium of a set
of markers or association between two marker blocks. More details
can be found in Zhao et al. (2000), Zhao and Sham (2002, 2003).

\medskip\noindent For example, to obtain p value of marker-disease
association based on 10,000 replicates of cases and controls in
4snps.dat, we can alter 4snps.par to be as follows,

\begin{verbatim}
4 1 0 10000  << ---- This has been changed
2 2 2 2
0 0
1 1 1 1
0 0 0 0
0.001 0.05 0.2 0.8
\end{verbatim}

\medskip\noindent GENECOUNTING/PERMUTE can be run as follows,

\medskip
{\bf gcp 4snps.par 4snps.dat 4snps.out}
\medskip

\medskip\noindent For both case-control and marker-marker analyses, the
program also yields haplotype specific tests.


\section{Other utilities}

\medskip\noindent It would be useful to extract haplotype assignment to be
used by other programs, which can be be done using the following
{\bf awk}. Suppose 4snps.awk has the following line

\begin{verbatim}
/\[1\]|\[2\]/{gsub(/\[|\]/,""); print;}
\end{verbatim}

\medskip\noindent On a Linux/Unix system we can extract the haplotype
assignment by

\medskip
{\bf awk -f 4snps.awk 4snps.out $>$ assign.dat}
\medskip

\noindent It can also be done similarly Under Windows.


\section{How to cite}

\medskip\noindent Please cite the following references if you use the
program in a publication.

\medskip\noindent
Zhao, J. H., Lissarrague, S., Essioux, L. and P. C. Sham (2002).
GENECOUNTING: haplotype analysis with missing genotypes.
Bioinformatics 18(12):1694-1695

\medskip\noindent
Zhao, J. H. (2004) 2LD, GENECOUNTING and HAP: Computer programs for linkage
disequilibrium analysis. Bioinformatics, 20:1325-1326


\section{Acknowledgement}

\medskip\noindent Thanks to Dr Sebastien Lissarrague for providing the SNP
data, and to Dr Andrew Pakstis for providing HAPLO result of
4snps.dat as comparison during the program development.


\section{References}

\medskip
\noindent
Zhao, J. H., Curtis, D. and Sham, P. C. (2000). Model-free analysis
and permutation test for allelic associations. Hum Hered 50:133-139

\medskip
\noindent
Zhao, J. H. and P. C. Sham (2002). Faster allelic association analysis using
unrelated subjects.  Hum Hered 53(1):36-41

\medskip
\noindent
Zhao, J. H. and P. C. Sham (2003). Generic number systems and haplotype
analysis. Comp Meth Prog Biomed 70: 1-9

\medskip
\noindent Zhao, J. H. (2007). gap: Genetic analysis package. J Stat
Soft 23(8):1-18


\section{Contact information}

\medskip\noindent
Jing Hua Zhao\\
MRC Epidemiology Unit\\
Institute of Metabolic Science\\
Box 285\\
Addenbrooke's Hospital, Hills Road\\
Cambridge CB2 0QQ\\
United Kingdom\\

\noindent Tel: +44 (0)1223 769165\\
email: \href{mailto:jinghua.zhao@mrc-epid.cam.ac.uk}{jinghua.zhao@mrc-epid.cam.ac.uk}\\
Web: \url{http://www.mrc-epid.cam.ac.uk/~jinghua.zhao/}

\bigskip\noindent {\small Date of last change 04-04-2008 by Jing Hua Zhao}
\end{document}
