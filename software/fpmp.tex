\documentclass[11pt]{article}
\begin{document}
\pagestyle{myheadings}
\markboth{FASTEHPLUS documentation}{FASTEHPLUS documentation}
\thispagestyle{empty}
\noindent{\large Documentation for FASTEHPLUS}

%13-6-2000, 3-4-2002, 23-5-2003

\medskip\noindent 1999-2003 Jing Hua Zhao

\section{Program description}

FASTEHPLUS performs model-free analysis and permutation test(s) of allelic
association based on EH (Xie and Ott 1993) and EHPLUS (Zhao et al.  2000).  It
uses marker genotypes from a group of unrelated individuals or a group of cases
and a group of controls and employs gene-counting algorithm to estimate
haplotype frequencies and output asymptotic and permutation test statistics.
As yet it does not handle missing genotypes. Please check for GENECOUNTING
program on this site if you wish to use individuals with missing genotypes
in your analysis.

Assume $m$ loci are involved each with ai alleles, $i=1,\dots,m$, they can
form $a_1 \times a_2 \times ...  \times a_m$ haplotypes.  Two hypotheses
can be considered.

\medskip
$H_0$: no association between markers;

\medskip
$H_1$: there is marker-marker association.
\medskip

Under $H_0$, haplotype frequencies can be obtained from product of their
constituent allele frequencies, so that the number of model parameters are
simply $N_0=(a_1-1)+(a_2-1)+...(a_m-1)$.  Under $H_1$, the haplotype
frequencies are obtained from gene counting method, a particular form of EM
algorithm.  The number of parameters becomes $N_1=a_1 \times a_2 \ldots, \times
a_m - 1$.  Denote the log-likelihoods under both assumptions as $\ln L_0$ and
$\ln L_1$, the log-likelihood ratio test statistic $2(\ln L_1-\ln L_0)$
asymptotically has $\chi^2$ distribution with $N_1-N_0$ degrees of freedom.

A block of markers in the data can be assumed to be associated but not the
others.  The marker block could either be within a chromosome segment or
any particular combination of typed markers.  Let log-likelihoods from
block 1 be $\ln L_0'$, $\ln L_1'$ and those from block 2 be $\ln L_0''$,
$\ln L_1''$, and their numbers of parameters as $N_0'$, $N_1'$, $N_0''$,
$N_1''$, then log-likelihood ratio test of two block association can be
specified as $2(\ln L_1-\ln L_1'-\ln L_1'')$ with $(N_1-N_1'-N_1'')$
degrees of freedom.

When a group of cases and a group of controls are involved, the program outputs
heterogeneity statistic, defined as

\medskip
$T_5 = -2(\ln L_1[cases+controls]-\ln L_1[cases]-\ln L_1[controls])$

\medskip
\noindent with labels in the brackets indicating sources of data, and the
statistic can be referred to a $\chi^2$ distribution with $N_1$ degrees of
freedom.

Since potentially there may be many haplotypes involved and asymptotic
approximation is likely to be unreliable, the program also performs empirical
inference via permutation tests.  A large number of replicates is generated by
randomly shuffling marker data or case-control labels.  This should break up
any hidden association between markers or between markers and the putative
disease locus as considered by the original EH.  Statistics from these
replicates then constitute an empirical distribution.  The location of the
observed statistic in this distribution provides empirical evidence for allelic
association.  In practice this is achieved by calculating proportion of
replicates that produce values of statistics at least as large as the observed.

When only marker data are involved, FASTEHPLUS considers three kinds of
permutations:  permuting every marker loci in the data, permuting block 1 while
keeping the second block 2 intact, and reporting result of block 1 only after
permutation.  For case-control data, only the observed case-control labels need
to be permuted.

Replicate statistics from permutation procedure can be used to measure linkage
disequilibrium (Zhao et al.  1999).  The permutation-based LD measure is
denoted $\xi$, and its sample estimate is denoted as $\hat\xi$, and $\hat\xi=
{\sqrt{2f}}({(t-\mu)}/{\sigma})/{N}$, where $t$ is the log-likelihood ratio
test statistic from the observed data, $f$ its degrees of freedom and $N$ the
number of individuals in the sample.  The mean ($\mu$) and variance
($\sigma^2$) of the likelihood ratio test statistic are based on its empirical
distribution obtained by permutation.  The sample variance of $\hat\xi$,
$2(f+2N\hat\xi)/N^2$, can be used to construct confidence interval.

\section{Input and output}

FASTEHPLUS needs two input files, a data file containing individual's ID,
affection status (group identity) with original genotyping, and a parameter
file describing these data.

\subsection{Data file}

The data file is a list of records containing individual's ID, affection status
(0=unaffected, 1=affected) and marker genotypes either in the format of

\begin{verbatim}
       [ID] [label]  [1a]   [1b]   [2a]   [2b] ...
 or
       [ID] [label]  [1]    [2] ...
\end{verbatim}

\noindent where [ID] and [label] are the individual's ID and case-control status
respectively.  For case-control analysis (specified in parameter file) [label]
takes values of 1 for cases and 0 for controls.

For the first format, columns [1a], [1b], [2a], [2b] are pairs of numbered
alleles at each marker separated by spaces.  For the second format columns
[1], [2], etc.  are genotype identifiers calculated from $(L+U(U-1)/2$,
where $L$, $U$, $L\leq U$ are the actual alleles of a specific marker.
For example with biallelic marker genotypes 1/1, 1/2 and 2/2 then the
genotype identifiers are 1, 2 and 3.

\subsection{Parameter file}

The parameter file contains control information such as number of marker loci,
type of analysis, number of permutation, and marker block, etc. specified in
six lines.

{\small
\medskip\noindent
line 1: \#1, \#2, \#3, \#4\\
line 2: alleles1, ..., alleles\#1\\
line 3: *1, *2\\
line 4: selected1, ..., selected\#1\\
line 5: permuted1, ..., permuted\#1\\
line 6: $q, f_0, f_1, f_2$\\
}

\noindent Line 1 specifies four numbers:  \#1, number of loci (i.e., $m$),
\#2, type of analysis (0=marker-marker analysis, 1=case-control analysis),
\#3, label-permutating indicator for case-control analysis and \#4, number
of permutations.  If ``type of analysis'' is specified to be 0 for a
case-control data a marker-marker analysis will be performed for cases and
controls combined together.

\medskip
\noindent Line 2 specifies number of alleles for all loci (i.e., $a_1,
\ldots, a_m)$ as indicated in line 1.

\medskip
\noindent Line 3 specifies whether the original marker data is single genotype identifier
(*1=1) or actual alleles (*1=0), and whether to output the identifier to screen
(0=no, 1=yes).

\medskip
\noindent Line 4 specifies marker selection status for each locus in the analysis, i.e.,
to be used in the analysis if its value is 1, not used if its value is 0.

\medskip
\noindent Line 5 specifies marker permutation status:  those taking values of 1 formed
one block and to be permuted; those taking values of 0 formed the other.  This
option is only for marker-marker analysis.

\medskip
\noindent Line 6 specifies the disease model, only kept to be compatible
with EHPLUS, i.e., specified but not used.  These are frequency of disease
allele, $q$, and three penetrances $f_0, f_1 ,f_2$ indicating
probabilities of being affected given there are 0, 1, 2 disease alleles at
the putative disease locus.  Note in the original EH penetrances for
disease genotypes are prompted as +/+, D/+ and D/D, where + and D
represent normal and disease alleles at a putative disease locus,
respectively.


\subsection{Output}

General information about data file and program control is reiterated on the
computer screen.  When only one analysis is requested all-subset analyses will
be conducted.

For marker-marker analysis, one block association is just as if an ordinary
statistic obtained from an EH-type analysis.  The other two statistics are as
described above.

For case-control data, statistics are given for cases only, controls only, and
heterogeneity statistics for all subsets of markers.  The main interests are
the heterogeneity statistic, especially when only one marker is involved in a
subset.

When replicate analysis is indicated, empirical $p$ values, $\hat\xi$ (xihat)
and its standard error are also calculated.  However this estimate for
heterogeneity statistic of case-control data is not so straightforward compared
to that of marker-marker analysis.

\section{Running the program}

MicroSoft Windows users will have to enter MS-DOS Prompt first.  For
example with WIN9x the operations will be

\medskip
Click Start $\rightarrow$ Select Programs $\rightarrow$ Select MS-DOS Prompt.

\medskip
Then change to the directory where FASTEHPLUS locates by {\bf cd} command (use
{\bf cd /?} to obtain more information).

The parameter, data and output files are supplied as command-line arguments to
FASTEHPLUS.  Optionally the program also reads a random number seed in place of
the default value 3000 for permutation test(s).  In other words, the syntax of
command is as follows.

\medskip
{\bf fpmp} $<$parameter file$> <$data file$> <$output file$>$ [seed]

\medskip
Those in angled brackets ($<>$) are compulsory, i.e., they need explicitly
specified, whereas the random number seed in squared bracket is optional, i.e.,
it may or may not be specified.


\section{Example: Association of alcoholism and ALDH2}

File aldh2.dat contains data of 130 Japanese alcoholics and 136 controls as
reported in Koch et al.  (2000).  Six microsatellite markers and two single
nucleotide polymorphisms (SNP) in the ALDH2 region (D12S2070, D12S839, D12S821,
D12S1344, EXON12, EXON1, D12S2263 and D12S1341) were genotyped.  They have
alleles 8, 8, 13, 14, 2, 2, 13 and 10 in the sample.

Apart from the subject ID at column 1, case-control indicators at column 2, it
simply a list of subjects with their marker genotypes at columns 3-18.

We now use data on two markers on either side of the functional locus EXON12.
They are conveniently numbered as 1, 2, 3 and 4.

File aldh2cc.par is created with the following lines.

\medskip
{\small\noindent
8  1  1  10000 $<<$ nloci, case/control, label permutation (1=permuted), npermute\\
8 8 13 14 2 2 13 10 $<<$ alleles\\
0 0          $<<$ is genotype, output genotypes (0=no, 1=yes)\\
0 0 1 1 0 1 1 0 $<<$ marker selection status (0=unselected, 1=selected)\\
0 0 0 0 0 0 0 0 $<<$ marker permutation status (0=not permuted, 1=permuted)\\
0.001 0.05 0.2 0.8 $<<$ disease model for case-control design\\
}

\noindent Line 1 has four numbers specifying aldh2.dat has 8 markers.  This is a
case-control analysis and permutation is conducted on affection status.
The number of permutations to be performed is 10,000.

\medskip
\noindent Line 2 lists the actual alleles at each marker mentioned above.

\medskip
\noindent Line 3 indicates that the marker genotypes in aldh2.dat are provided as pairs
of alleles and that the genotype identifiers will not be shown on the screen.

\medskip
\noindent Line 4 selects markers D12S821, D12S1344, EXON1 and D12S2263 for the analysis.

\medskip
\noindent Line 5 specifies markers are not to be permuted.

\medskip
\noindent Line 6 defines the disease model.  We can leave this line intact since we
currently only use heterogeneity statistic for case-control data.

\medskip
Now we specify

\medskip
{\bf fpmp} aldh2cc.par aldh2.dat aldh2cc.out
\medskip

to obtain output in file aldh2cc.out.

The heterogeneity $\chi^2$ statistic is 214.72 with degrees of freedom
4055 based on asymptotic approximation. We have to rely on permutation
tests.

The same dataset can be used to perform marker-marker analysis.  File
aldh2mm.dat is obtained by slight modification of aldh2cc.par.

{\small
\medskip\noindent
8  0  0  10000        $<<$ This line has been changed\\
8 8 13 14 2 2 13 10\\
0 0\\
0 0 1 1 0 1 1 0\\
0 0 1 1 0 0 0 0       $<<$ This line has been changed\\
0.001 0.05 0.2 0.8\\
}

Changes are made at the second number of line 1, which tells FASTEHPLUS to do
marker-marker analysis.  The fifth line has also been changed.  Since this is a
marker-marker analysis we use the option to see if any association between
block 1, containing D12S821, D12S1344 and block 2, containing EXON1, D12S2263.
We will use random number seed 50,000.

\medskip
Our command now becomes

\medskip
{\bf fpmp} aldh2mm.par aldh2.dat aldh2mm.out  50000
\medskip

The output will be written to aldh2mm.out.

As shown by these two analyses, we can modify our parameter file for any
desirable subset analysis and permutation test.  Multiple runs can be initiated
via DOS or Unix batch files.  Unless the original problem is small, this would
be computer-intensive.


\section{Notes on changes, program constants and compiling}

There is one noticeable difference in marker-marker analysis between EHPLUS and
FASTEHPLUS when blockwise association is examined:  the third statistic in
FASTEHPLUS is only for the permuted block.  For case-control analysis, it
suppresses statistics base on disease model.  It is possible to modify
FASTEHPLUS source code slightly to allow for case-control type analysis using
parametric model.  A single EH-type analysis could be achieved by
PREPFASTEHPLUS and FASTEHPLUS.  PREPFASTEHPLUS program uses input files for
FASTEHPLUS to generate an output file, to be used by FASTEHPLUS for ordinary EH
analysis.

For instance, to run a marker-marker analysis of the ALDH2 dataset we use the
following command:

\medskip
{\bf pfehp} aldh2mm.par aldh2.dat aldh2mm.dat

\medskip
{\bf fehp}

\medskip
and answer the queries as follows

Do you wish to use the case-control sampling option?  [N]

Enter name of data file  [EHPLUS.DAT]

aldh2mm.dat

you entered: aldh2mm.dat

Enter name of output file.  [EHPLUS.OUT]

aldh2mm.out

\medskip
The result is then stored in file aldh2mm.out.


A full list of the distributed files is given as follows.

\medskip
{\small
\begin{tabular}{ll}
\hline
File name  & Description\\
\hline
fehp.c     & FASTEHPLUS single analysis source file\\
fehp.h     & FASTEHPLUS single analysis header file\\
fehp.exe   & FASTEHPLUS single analysis executable file\\
fpmp.c     & FASTPMPLUS, permutation/model-free analysis source file\\
fpmp.h     & FASTPMPLUS, permutation/model-free analysis header file\\
fpmp.exe   & FASTPMPLUS, permutation/model-free analysis executable file\\
fpmp.doc   & Documentation in ASCII format\\
fpmp.pdf   & This file\\
pfehp.c    & PREPFASTEHPLUS source file\\
pfehp.h    & PREPFASTEHPLUS header file\\
pfehp.exe  & PREPFASTEHPLUS executable\\
aldh2.dat  & ALDH2 data file\\
aldh2cc.par& ALDH2 parameter file for case-control analysis\\
aldh2mm.par& ALDH2 parameter file for marker-marker analysis\\
\hline
\end{tabular}
}

\medskip
Program fpmp.c integrates the functionality of EHPLUS programs ehplus.c and
pmplus.c.  Program fehp.c is a faster version of ehplus.c for a single analysis
without permutation, and allows for disease model of a putative disease locus
to be specified (Xie and Ott 1993; Ott 1998).  Program pfehp.c prepares input
file for fehp.c.

Two program constants are maximum number of loci (MAX\_LOC) and maximum number
of alleles (maxalleles) at a locus, set to MAX\_LOC=30, maxalleles=50.  To alter
them simply locate them in fpmp.h and change to the desired values.

Examples for building WIN9x/NT or Unix executables are as follows.

\medskip
Borland/Inprise C

\medskip
{\bf bcc} -Iinclude -Llib -mh fpmp.c
\medskip

or

\medskip
{\bf bcc32} -Iinclude -Llib fpmp.c

assuming current directory contains include and lib subdirectories for C header
and library files.

\medskip
Symantec C

\medskip
{\bf sc} -mn fpmp.c

\medskip
assuming include and lib subdirectories are properly set.

\medskip
MicroSoft Visual C

\medskip
{\bf vcvars32}

{\bf cl} fpmp.c

\medskip
Appropriate environments can be set via runing batch file VCVARS32.

\medskip
Cygwin gcc
\medskip

{\bf gcc} fpmp.c -o fpmp

\medskip
Unix gcc

\medskip
{\bf gcc} fpmp.c -lm -o fpmp

\medskip
Under Unix, pfehp, fehp and fpmp can be created by a single command

\medskip
make


\section{Acknowledgement}

Thanks to Dr Wentian Li for providing makefile and Dr Dimitri Zaykin for
many suggestions.  Thanks also to many colleagues for comments and program
testing.


\newpage\section{Contact information}

If you have any questions, comments and suggestions, please contact me
via {\bf e-mail} j.zhao@public-health.ucl.ac.uk, or by post to

\medskip\noindent
Jing Hua Zhao\\
Department of Epidemiology \& Public Health\\
University College London\\
1-19 Torrington Place\\
London WC1E 6BT\\
The United Kingdom\\
Tel +44 (0)20 7679 5627


\section{How to cite}

Zhao JH and Sham PC (2002) Faster allelic association analysis using unrelated
subjects.  Hum Hered, 53(1):36-41


\section{References}

\medskip
\noindent
Koch HG, McClay J, Loh E-W, Higuchi S, Zhao J-H, Sham P, Ball D, et al (2000)
Allele association studies with SSR and SNP markers at known physical distances
within a 1 Mb region embracing the ALDH2 locus in the Japanese, demonstrates
linkage disequilibrium extending up to 400 kb.  Hum Mol Genet 9:2993-2999

\medskip
\noindent
Ott J. (1998) User's Guide to EH. http://linkage.rockefeller.edu

\medskip
\noindent
Xie X.  and J.  Ott (1993):  Testing linkage disequilibrium between a disease
gene and marker loci.  Am J Hum Genet 53:1107

\medskip
\noindent
Zhao JH, Curtis D, Sham PC (2000) Model-free analysis and permutation tests for
allelic associations. Hum Hered 50:133-139

\medskip
\noindent
Zhao H, Pakstis AJ, Kidd JR, Kidd KK (1999) Assessing linkage disequilibrium in
a complex genetic system. I. Overall deviation from random association. Ann
Hum Genet 63:167-179
\end{document}

