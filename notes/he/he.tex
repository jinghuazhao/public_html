\documentstyle[12pt]{article}
% change \thetable to Roman in article.cls
\begin{document}
{\noindent \huge\bf Class Discussion:}\bigskip

Haseman, J.K. and R.C. Elston, 1972, ``The investigation of linkage
between a quantitative trait and a marker locus'', Behavior Genetics, Vol.
2, No. 1, 3-19.

\section{An Overview}

  Haseman and Elston (1972) paper on quantitative traits was among the
substantial developments in the 1970s.  As of today there are many extensions
of the method; it is thus worthy to review it for some fundamental insight what
the present methodologies work.  From these understanding we conceive that
indeed the recent plethora about sib-pair analysis does not happen overnight.

Here I would highlight some points of its formulation and
its practical applications. That paper dealt with QTL, however, the
topic itself does not restrict us to quantitative trait only.

\section{The main material}

\subsection{The problem}
  It's not a easy matter in man to establish a trait with a genetic
component. Different existing methods all have their drawbacks. Here are the
points made in the paper.
\begin{itemize}
\item Twin studies tends to overestimate the genetic component (H-E
1970).
\item Classical segregation analysis only apply to qualitative trait.
\item Linkage analysis of trait and marker would be very helpful,
especially when increasing number of marker loci available in man.
The focus on sib pairs avoid secular or age effects. Penrose first
considered the sib pair data, but this paper allowed for
incorporation of parental information on marker loci. The
recombination fraction was also addressed.
\end{itemize}
While parametric linkage analysis depends heavily on the model specifications
(e.g., the penetrance) and has difficulty in collecting large pedigrees,
sib-pair analysis requires no such assumptions and thus more robust
and easy to implement in terms of its design and computation.  The
drawback might be a larger sample size is necessary.  In general it is a very
powerful strategy comparable to association analysis, TDT and parametric linkage
analysis.
\subsection{Some terms}
\begin{itemize}
\item{IBD/IBS.} Any two copies of allele 1 at a given locus are
considered to be identical by state (IBS), but only copies from
allele 1 that are inherited from a common ancestral source are said
to be identical by descent(IBD). If two alleles are IBD, they are
also IBS.
\item{Linkage equilibrium.} It refers to the situation where genes segregate
independently. If linkage disequilibrium occurs we also term it as allelic association.
\item{Genetic heterogeneity.} Any of a number of genetic causes can act
independently to produce an identical disease phenotype.
\item{Phenocopy.} The phenotype is simply a copy from environment. In model
specification this means we have non-zero penetrances for all genotypes.
\end{itemize}

\subsection{The genetic model}
The model is $$  x=\mu + g + e$$
where $\mu$ is the overall mean, $g$ is a major gene effect, and $e$ is an
environmental effect independent of $g$.  $g$ is $a$, $d$ and $-a$ for unobserved
genotype BB, Bb and bb.  $E(g)=E(e)=0$.

To apply the above equation to sib pair we have $x_{1j}=\mu +
g_{1j}+e_{1j}$, $x_{2j}=g_{2j}+e_{2j}$ From these we could expect a relation
for the squared pair difference $Y_j={(x_{1j}-x_{2j})}^2$ of sib pair $j$,
under any possible combinations as listed in H-E table I.

\begin{table}[h]\centering
\caption{Conditional Distribution of $Y_j$\label{table1}}
\begin{tabular}{lclll}
\hline
&&\multicolumn{3}{c}{Conditional probability}\\ \cline{3-5}
Sib pair& $Y_j$& $\pi_j=0$ & $\pi_j=\frac{1}{2}$ & $\pi_j=1$\\
\hline
BB-BB &$e_j^2$ & $p^4$ & $p^3$ & $p^2$\\
bb-bb &$e_j^2$ & $q^4$ & $q^3$ & $q^2$\\
Bb-Bb &$e_j^2$ & $4p^2q^2$ & $pq$ & $2pq$\\
BB-Bb &$(a-d+e_j)^2$ & $2p^3q$ & $p^2q$ & 0\\
Bb-BB &$(-a+d+e_j)^2$&$2p^3q$ &$p^2q$ & 0 \\
Bb-bb &$(d+d+e_j)^2$&$2pq^3$ &$pq^2$ & 0 \\
bb-Bb &$(-a-d+e_j)^2$&$2pq^3$ &$pq^2$ & 0 \\
BB-bb &$(2a+e_j)^2$&$p^2q^2$ &$0$ & 0 \\
bb-BB &$(-2a+e_j)^2$&$p^2q^2$ &$0$ & 0 \\
\hline
\end{tabular}
\end{table}

\begin{itemize}
\item $\pi_j=0$ signifies
sibs are "unrelated" at the trait locus, so the distribution simply follows the
usual Hardy-Weinberg Law of random-mating population.
\item $\pi_j=1$ when both sibs have the same genotype so the probability is just
the probability in population of one of them.
\item $\pi_j={1\over2}$ is similarly reasoned yet more complicated. 
\end{itemize}

From table~\ref{table1}, we can have the expected value of $Y_j$ given $\pi_j,
j=0,\frac{1}{2}, 1$, i.e.,
$$E[Y_j|\pi_j]=\sigma_e^2+2\sigma_a^2+2\sigma_d^2,\quad 
\sigma_e^2+\sigma_a^2+2\sigma_d^2,\quad
\sigma_e^2$$
and in a more general case
$$E[Y_j|\pi_j]=(\sigma_e^2+2\sigma_g^2)-\pi_j[2\sigma_g^2+2n_1(n_2-n_0)\sigma_d^2/(4n_0n_2+n_0n_1+n_1n_2)]$$

Assuming no dominance ($d=0$ or $\sigma_d=0$) we see that
$E(Y_j|\pi_j)=\alpha+\beta\pi_j, \alpha={\sigma_e}^2+2{\sigma_g}^2$, and
$\beta=-2{\sigma_g}^2$. 

In a more general formulation, a term for polygenic effect and/or covariates
may also be included.  It is familiar to see this for variance components/path
analysis models.

In derivation of the variance of a quantitative trait, we take BB, Bb and bb as
$2\alpha, \alpha, 0$ respectively, by considering that the variance is
independent of origin.  Thus in a random-mating population, the mean and
variance (genetic variance of the population) would be $2p\alpha$ and
$2pq{\alpha}^2$.  In many situations it is convenient to take genic effect as
unit of measurement, then g has values 2, 1, 0 and the mean and variance change
to $2p$ and $2pq$.

To determine the effect of dominance on the variance, we fit a set of
hypothetical G values for which the effect of each gene substitution is constant.
The phenotypic and fitted values then have relationship
${\sigma_x}^2={\sigma_a}^2+{\sigma_d}^2$ The fit of $X=G+D$, $D$ has forms
$D_{BB}, D_{Bb}$ and $D_{bb}$, is by minimizing

  $$\Delta = \Sigma fD^2=p^2{D_{bb}}^2+2pq{D_{Bb}}^2+q^2{D_{bb}}^2$$
then
  $$D_{BB}=-q^2(h_1 -h_2)$$
  $$D_{Bb}=pq(h_1 - h_2) $$
  $$D_{bb}=-p^2(h_1 -h_2)$$
  where $h_1=x_{Bb}-x_{bb}, h_2=x_{BB}-x_{Bb}$ and $\alpha$ could be obtained
see, Li(1955).

  $${\sigma_a}^2=2pq{\alpha}^2=2pq{(ph_2+qh_1)}^2, {\sigma_D}^2=p^2q^2{(h_1 -h_2)}^2$$

We have

  $${\sigma_a}^2=2pq{[a-d(p-q)]}^2\quad  {\sigma_d}^2=4p^2q^2d^2$$

\subsection{Including marker information}

   Let $f_{ji}$ be the probability that $j^{th}$ sib pair has $i$ alleles IBD at the
marker locus, $i=0,1,2$ conditional on the nuclear family data available, then $\pi_j$,
the proportion of alleles the $j^{th}$ sib pair shares IBD at the marker locus, is given
by $\hat{\pi}_j=f_{j2}+{1\over2}f_{j1}$.  The proportion of alleles the $j^{th}$
half-sib pair share IBD at the marker locus is estimated by
$\hat{\pi}_j={1\over2}f_{j1}$. 

   With half-sibs, one schematic form is as follows.

\begin{center}
\begin{tabular}{rcccl}
           \fbox{1}& $\rightarrow\leftarrow$ &(2)&  $\rightarrow\leftarrow$ &\fbox{3} \\
               &$\downarrow$ && $\downarrow$ & \\
               &\fbox{$S_1$}  &&  \fbox{$S_2$}&\\
\end{tabular}
\end{center}

  It would display a 'cluster' format (SAGE/SIBPAL) if any parent
has more than three spouses and then another kinds of sibs.

  The same reasoning process applies to the estimation involving
marker information, i.e., the notion of mating type and conditional
probability, as shown in E-H table II.

\begin{table}[h]\centering
\caption{$\hat\pi_j$ when both parental and sib genotypes are known\label{table2}}
\begin{tabular}{rlrcccccc}
\hline
\multicolumn{2}{c}{Mating type} & 
\multicolumn{2}{c}{Sib pair type} & Probability & $f_{j0}$ & $f_{j1}$ & $f_{j2}$ & $\hat\pi_j$ \\
\hline
I: & $A_iA_i\times A_iA_i$ & I: & $A_iA_i-A_iA_i$ & $p_i^4$ & $\frac{1}{4}$ & $\frac{1}{2}$ & $\frac{1}{4}$ & $\frac{1}{2}$ \\
II: & $A_iA_i\times A_jA_j$ & V: & $A_iA_j-A_iA_j$ & $2p_i^2p_j^2$ & $\frac{1}{4}$ & $\frac{1}{2}$ & $\frac{1}{4}$ & $\frac{1}{2}$ \\
III: & $A_iA_i\times A_iA_j$ & I: & $A_iA_i-A_iA_i$ & $p_i^3p_j$ & 0 & $\frac{1}{2}$ & $\frac{1}{2}$ & $\frac{3}{4}$ \\
                             && III: & $A_iA_i-A_iA_j$ & $2p_i^3p_j$ & $\frac{1}{2}$ & $\frac{1}{2}$ & 0 & $\frac{1}{4}$ \\
                             && V: & $A_iA_j-A_iA_j$ & $p_i^3p_j$ & 0 & $\frac{1}{2}$ & $\frac{1}{2}$ & $\frac{3}{4}$ \\
IV: & $A_iA_i\times A_jA_k$ & V: & (2) & $p_i^2p_jp_k$ & 0 & $\frac{1}{2}$ & $\frac{1}{2}$ & $\frac{3}{4}$ \\
                            && VI: & $A_iA_j\times A_iA_k$ & $2p_i^2p_jp_k$ & $\frac{1}{2}$ & $\frac{1}{2}$ & 0 & $\frac{1}{4}$ \\
V: & $A_iA_j\times A_iA_j$ & I: & (2) & $p_i^2p_j^2/4$ & 0 & 0 & 1 & 1 \\
                            && II: & $A_iA_i\times A_jA_j$ & $p_i^2p_j^2/2$ & 1 & 0 & 0 & 0 \\
                            && III: & (2) & $p_i^2p_j^2$ & 0 & 1 & 0 & $\frac{1}{2}$ \\
                            && V: & $A_iA_j\times A_iA_j$ & $p_i^2p_j^2$ & $\frac{1}{2}$ & 0 & $\frac{1}{2}$ & $\frac{1}{2}$ \\
VI: & $A_iA_j\times A_iA_k$ & I: & $A_iA_i-A_iA_i$ & $p_i^2p_jp_k/2$ & 0 & 0 & 1 & 1 \\
                          &&III: & (2) & $p_i^2p_ip_k$ & 0 & 1 & 0 & $\frac{1}{2}$ \\
                           &&IV: & $A_iA_i-A_jA_k$ & $p_i^2p_jp_k$ & 1 & 0 & 0 & 0 \\
                      &&V: & (3) & $p_i^2p_jp_k/2$ & 0 & 0 & 1 & 1 \\
                           &&VI: & $A_iA_j-A_iA_k$ & $p_i^2p_jp_k$ & 1 & 0 & 0 & 0 \\
                           &&VI: & ${A_iA_j-A_jA_k} \atop {A_iA_k-A_jA_k}$ & $p_i^2p_jp_k$ & 0 & 1 & 0 & $\frac{1}{2}$ \\
VII: & $A_iA_i\times A_kA_l$ & V: & (4) & $p_ip_jp_kp_l/2$ & 0 & 0 & 1 & 1 \\
                            &&VI: & (4) & $p_ip_jp_kp_l$ & 0 & 1 & 0 & $\frac{1}{2}$ \\
                           &&VII: & (2) & $p_ip_jp_kp_l$ & 1 & 0 & 0 & 0 \\
\hline
\end{tabular}
\end{table}

  With any genotypic information unknown, $f_{ji}$ would be more difficult to obtain, an
estimation needs to be done from the parent and offspring phenosets. 

   $$f_{ji}=\frac{\sum_{\nu\in P_P} \sum_{w\in P_S} P(\nu {~\verb|and|~} w{~\verb|and|~}\pi_j={i\over 2})}
 {\sum^2_{h=0} \sum_{\nu\in P_P} \sum_{w\in P_S} P(v{~\verb|and|~}w{~\verb|and|~}\pi_j ={h\over 2})}$$
where $i=0,1,2$.

  The numerator is the joint probability of observing $I_m$ and that $\pi_j$
equals $i\over2$; the denominator is the sum of the three such joint
probability for $i=0,1,2$.  Results for no dominance and no parental
information is given in their table III (here $\Phi=(1+p_i+p_j+2p_ip_j)$).

\begin{table}\centering
\caption{$\hat\pi_j$ when there is no dominance and parental genotypes are unknown\label{table3}}
\begin{tabular}{rlccccc}
\hline
\multicolumn{2}{c}{Sib pair type} & Probability & $f_{j0}$ & $f_{j1}$ & $f_{j2}$ & $\hat\pi_j$ \\
\hline
I: & $A_iA_i\times A_iA_i$ & $p_i^2(1+p_i)^2/4$ & $\frac{p_i^2}{(1+p_i)^2}$ & $\frac{2p_i}{(1+p_i)^2}$ & $\frac{1}{(1+p_i)^2}$ & $\frac{1}{(1+p_i)}$ \\
II: & $A_iA_i\times A_jA_j$ & $p_i^2p_j^2/2$ & 1 & 0 & 0 & 0 \\
III: & $A_iA_i\times A_iA_j$ & $p_i^2p_j(1+p_i)$ & $\frac{p_i}{1+p_i}$ & $\frac{1}{1+p_i}$ & 0 & $\frac{1}{2(1+p_i)}$ \\
IV: & $A_iA_i\times A_jA_k$ & $p_i^2p_jp_k$ & 1 & 0 & 0 & 0 \\
V: & $A_iA_j\times A_iA_j$ & $p_ip_j\Phi/2$ & $\frac{2p_ip_j}{\Phi}$ & $\frac{p_i+p_j}{\Phi}$ & $\frac{1}{\Phi}$ & $\frac{2+p_i+p_j}{2\Phi}$ \\
VI: & $A_iA_j\times A_iA_k$ & $p_ip_jp_k(1+2p_i)$ & $\frac{2p_i}{1+2p_2}$ & $\frac{1}{1+2p_i}$ & 0 & $\frac{1}{2(1+2p_i)}$ \\
VII: & $A_iA_j\times A_kA_l$ & $2p_ip_jp_kp_l$ & 1 & 0 & 0 & 0 \\
\hline
\end{tabular}
\end{table}


\subsection{Other aspects}
  The null hypothesis for usual linkage analysis is $H_0: c=0; H_1={1\over2}$.
  It turns out simple to bear that the test is $H_0:slope=0$ vs $H_1:
slope<0$  where linkage is expected.
\begin{eqnarray*}
E[Y_j|\hat{\pi}_{jm}]&=&\sum_{\pi_{jk}}E(Y_j|\pi_{jt})P(\pi_{jt}|\hat{\pi}_{jm}) \\
&=&\sum_{\pi_{jt}}\sum_{\pi_{jm}}E(Y_j|\pi_{jt})P(\pi_{jt}|\pi_{jm})P(\pi_{jm}|\hat{\pi}_{jm}) \\
\end{eqnarray*}
The first part simply combines all the possibilities
given marker data while the second part relates to a clever inclusion of marker
probability.

\begin{table}[h]\centering
\caption{Joint distribution of $\pi_{jm}$ and $\pi_{jt}$\label{table4}}
\begin{tabular}{ccccc}
\hline
&\multicolumn{3}{c}{$\pi_{jm}$} \\ \cline{2-4}
$\pi_{jt}$&0 & $\frac{1}{2}$ & 1 & Total \\
\hline
0  & $\Psi^2/4$ & $\Psi(1-\Psi)/2$ & $(1-\Psi)^2/4$ & $\frac{1}{4}$\\
$\frac{1}{2}$ & $\Psi(1-\Psi)/2$ & $(1-2\Psi+2\Psi^2)/2$ & $\Psi(1-\Psi)/2$ & $\frac{1}{2}$ \\
1  & $(1-\Psi)^2/4$ & $\Psi(1-\Psi)/2$ & $\Psi^2/4$ & $\frac{1}{4}$ \\
Total &$\frac{1}{2}$& $\frac{1}{2}$& $\frac{1}{4}$& 1\\
\hline
\end{tabular}
\end{table}

  It deserves to note the way recombination fraction c was included
in the model described in their appendix B. The joint distribution of
$\pi_{jm}$ and $\hat\pi_{jm}$ can be derived from table II. For the
special case of a two-allele marker gene this was given in table V.

\begin{table}[h]\centering
\caption{Joint distribution of $\hat\pi_{jm}$ and $\pi_{jt}$ for a
two-allele marker locus with no dominance and complete parental
information \label{table5}}
\begin{tabular}{ccccc}
\hline
&\multicolumn{3}{c}{$\pi_{jm}$} \\ \cline{2-4}
$\hat\pi_{jm}$&0 & $\frac{1}{2}$ & 1 & Total \\
\hline
0  & $\frac{1}{2}p^2q^2$ & 0 & 0 & $\frac{1}{2}p^2q^2$\\
$\frac{1}{4}$ & $p^3q+pq^3$ & $p^3q+pq^3$ & 0 & $2(p^3q+pq^3)$ \\
$\frac{1}{2}$ & $\frac{1}{4}(p^4+4p^2q^2+q^4)$ & $\frac{1}{2}(p^4+6p^2q^2+q^4)$ &
$\frac{1}{4}(p^4+4p^2q^2+q^4)$ & $(p^4+5p^2q^2+q^4)$ \\
$\frac{3}{4}$ & 0 & $p^3q+pq^3$ & $p^3q+pq^3$ & $2(p^3q+pq^3)$ \\
1  & 0 & 0 & $\frac{1}{2}p^2q^2$ & $\frac{1}{2}p^2q^2$ \\
Total &$\frac{1}{2}$& $\frac{1}{2}$& $\frac{1}{4}$& 1\\
\hline
\end{tabular}
\end{table}

  Note the conditional probabilities of $p(\pi_{jt}|\pi_{jm})$ and
$p(\pi_{jm}|\hat{\pi}_{jm})$: the first elements of $p(\pi_{jt}|\pi_{jm})$ is

  ${\Psi^2/4}\over {1/4}$,$ {{\Psi (1-\Psi)/2}\over{1/2}}$,
  ${{{{(1-\Psi)}^2}/4}\over{1/4}}$ and the other two elements are obtained
accordingly, here $\Psi=c^2+{(1-c)}^2$. Formulation of
$p(\pi_{jm}|\hat{\pi}_{jm})$ is similar.

  For $E[Y_{jt}|\hat{\pi}_{jm}=1]$, it is
  $$\sigma^2_e\Psi^2+[\sigma^2_e+\sigma^2_g]
[2\Psi(1-\Psi)]+[\sigma^2_e+2\sigma^2_g][{(1-\Psi)}^2] $$

that equals to ${\sigma_e}^2+2(1-\Psi){\sigma_g}^2$.  The derivation of
$E(Y_{jt}|\hat{\pi}={1\over2})$ is a bit tedious.

  In summary, the expression for $E[Y_{jt}|\hat{\pi}_{jm}]$ is
  $$[{\sigma_e}^2+2(1-2c+2c^2){\sigma_g}^2]-2{(1-2c)}^2{\sigma_g}^2\hat{\pi}_{jm}
$$

  This simply has the form
  $$E[Y_j|\hat{\pi}_j]=\alpha+\beta \hat{\pi}_j$$
  where  $\beta=-2{(1-2c)}^2{\sigma_g}^2$.

  The test of the null hypothesis of no linkage turns out to be a
test of $\beta $
a significantly non-zero of $\beta$ implies linkage, i.e., $c\ne {1\over2}$

  Further, suppose that there are K trait loci, each linked to the
trait loci, $c$ be recombination fraction, they shows
  $$E[\hat{\beta}]=-2\sum_{i=1}^K {(1-2c_i)}^2{\sigma_i}^2$$
  where ${\sigma_i}^2$ is the component of total genetic variance of the
$i^{th}$ trait locus.

  The underlying assumption would be K trait loci are mutually
unlinked and there is no epistasis.

\subsection{Maximum likelihood estimation of linkage}
The proposed method has a disadvantage that $\sigma_g^2$ is confounded with
recombination fraction $c$ so that linkage could only be detected but not
estimated. The paper proceeds to assume sib pair differences are normally
distributed as a mixture of up to seven normal distributions. By considering
only the absolution pair differences the number of distributions could be 
reduced to four $f_1$,$f_2$,$f_3$,$f_4$. They showed that the likelihood 
function for a sib pair can be expressed as in terms of five parameters $c$, 
$p$, $\sigma_e^2$, and $d$,
$$L=f(D|I_m)=\sum\limits_{h=0}^2\sum\limits_{k=0}^2f(D_j|\pi_{jt}=h/2)
P(\pi_{jt}=h/2|\pi_{jm}=k/2)P(\pi_{jm}=k/2|I_m)$$
where $f(D_j|\pi_{jt}=h/2)=m_{h1}f_1+m_{h2}f_2+m_{h3}f_3+m_{h4}f_4$.
Values of $m_{hi}, \quad h=0,1,2;\quad i=1,2,3,4$ can be obtained from 
table I and were given in table VI.

\begin{table}[h]\centering
\caption{Values of the coefficient $m_{hi}$}
\begin{tabular}{cclll}
\hline
& 1&2&3&4\\
\hline
0 &$p^4+4p^2q^2+q^4$ & $4p^3q$ & $4pq^3$ & $2p^2q^2$\\
1 &$1-2pq$ & $2p^2q$ & $2pq^2$ & 0 \\
2 &1 & 0 & 0 & 0 \\
\hline
\end{tabular}
\end{table}


\section{Extensions} Like many other commonly used linkage procedures,
it assumes linkage equilibrium and no clinical and genetic
heterogeneity.

  In addition to notes above, one conceivable example is that we need to take
the effect of sibship size into considerations, as proposed by Hodge (1984) and
implemented in GAS (Genetic Analysis System).

Blackwelder and Elston (1985) and later Knapp et al (1994) described the power 
of various statistics of
affected sib pair analysis; Goldgar(1990) handles multipoint human data on
multifactorial traits but confounds the effect of the linked quantitative trait
loci with the total additive genetic variance.  H-E method was extended by Amos
and Elston (1989) to other types of relative pairs.  Amos(1994) discussed the
estimation of the recombination fraction from relative pair data using
likelihood and quasi-likelihood methods.

For highly polymorphic marker such as HLA, IBS information is also invaluable, see
Lange (1986).  For diseases with a late age of onset, many parental couples
will be unavailable for genic typing.  The method classifies affected sib pairs
based on IBS information as concordant, discordant and half-maker concordant
without referring to parental genotypes.  It's possible to compare the expected
and observed number of affected sib-pairs falling into each categories by
$\chi^2$ test. The method was also extended to consider multiple sibs.

Sib-pair interval mapping is a multi-point method in which information from
adjacent markers is used to infer missing or ambiguous allele sharing.
  $$x_i=\mu+g_i+G_i+\sum^s_{k=1}\beta_k z_{ik}+e_i$$
$z_{ik}$ is the kth covariate measurement on an individual. $g_i=a,d,-a$
for BB, Bb and bb, $E(g_i)=E(e_i)=E(G_i)=0$. $G_i$ is a random
polygenic effect. We have $E(x_i)=\mu+\sum^s_{k=1}\beta_k z_{ik}$ and
$COV(x_i,x_j)={\sigma_a}^2+{\sigma_d}^2+{\sigma_G}^2+{\sigma_e}^2$ for i=j and
$\Phi_{ij}{\sigma_a}^2+\Delta_{ij}\sigma^2_d+\Phi_{ij}\sigma^2_G$ for $i\ne j$,
where $\sigma^2_g$
and $\sigma^2_e$ are respectively, the polygenic and residual components of
variance.  and $\sigma^2_a$ and $\sigma^2_d$ as before.
$\Phi_{ij}$ is the coefficient of relationship between relatives, $\Delta_{ij}$
is the probability a pair shares both alleles at the major locus IBD.
With marker data available $E(x_i|\pi_{tij})$ and $COV(x_i,x_j|\pi_{tij})$ have
similar expressions. With linked markers,
$E(x_i-x_j|\pi_{ij})=2Var(x^2_i)-2cov(x_i x_j|\pi_{ij})$ , now
$cov(x_i,x_j|\pi_{ij})$
includes marker information. When joint distribution of  is multinormal,
then we have likelihood function to which optimization or other
estimation methods can be applied, see Amos (1994).

Affected sib-pair analysis (ASP) is more widely used in the linkage design and
analysis.  It avoids incomplete penetrance and supposes that if a given marker
is cosegregating with a disease- predisposing allele, then affected siblings of
affected persons are more likely to receive the same allele identical by
descent at a closely linked marker locus than if the marker locus was
segregating independently.

The Extended Sib pair analysis (ESPA) calculates a chi-square test statistic
$\chi^2=2{[S-E(S)]}^2/E(S), E(S)=[S+NS]/2$, $S$ and $NS$ are number of observed
allele shared/not shared IBD, of the null hypothesis of no linkage.  It employs
LINKAGE/MLINK program to estimate the possible marker genotypes for untyped
individuals.  Thus IBD is a probability count that may lead to bias.

In extended pedigree to examine IBS relationships, the so-called affected
pedigree member (APM) method.  The statistic is

$Z_{ij}={1\over4}\sum^2_{a=1}\sum^a_{b=1}\delta(A_a,B_b)f(A_a)$, where $f(A_a)$
is a weight function taking forms of $1,1/\sqrt{p_{A_a}}, 1/\sqrt{p_{A_a}}$,
the latter two give more weight to rare-allele sharing.

Other scheme may include study design, see for instance, Boehnke (1990) and
Risch and Zhang (1995).  Actually these papers considered the quantitative
traits in their extremity values.  To avoid the influence of allele
frequencies, affected sib-pair with parental information is attractive.

\section{Implementations}
Some aspects that have been considered:
\begin{itemize}
\item Information contained in multiple sibs
\item Complex sibship and pedigree structure
\item Estimation of possible genotype
\item Multiloci or haplotype, its implementation
\item Covariates
\end{itemize}
Here are a few softwares that are useful.
\begin{itemize}
\item{SAGE/SIBPAL} could handle both qualitative and binary trait with
or without variable age of onset. It can also take covariates.
\item{ESPA} is used for 'extended' sib-pair analysis.
\item{GAS/SIBHE} GAS 2.0 has graphic utility for PostScript file.  Among
others, GAS/SIBIHE combines interval mapping with the H-E algorithm. It also
uses Jeff O'Connell's Vitesse program.  It is able to handle up to 8 highly
polymorphic loci simultaneously.  It also has modules for IBD and Lange's IBS
analysis referred above.
\item{APM} may ignore information on relative haplotype and also
depends heavily on allele frequencies.
\end{itemize}

{\bf User control.}  GAS, MAPMAKER/SIBS provide user-friendly commands.
However most programs use LINKAGE/MAKEPED .PED format or .PRE.

\section{Applications} The method has wide application, such as serum IgE
level with interleukin-4 encoding gene, the angiotensinogen(IBS-APM) and
essential hypertension, Alzheimer's Disease(late-onset with chromosome 19).
Actually the basic theory has become one of the strategies for genetic
dissection of complex trait.  Affected sib-pair have played an important role
in the study of type I diabetes and HLA.

\subsection*{Acknowledgements}

In writing this, some of the descriptions are excerpts from the original papers
due to time limitations; some changes are made to consist with H-E's original
paper.

\section*{References}

\begin{enumerate}

\item Haseman, J.K.  and Elston, R.C.  (1970).  The estimation of genetic
variance from twin data.  BG 1:  11-20.

\item Blackwelder W.C.  and Elston, R.C.  (1985).  A comparison of sib-pair
linkage tests for disease susceptibility loci, GE 2:85- 97.

\item Amos, C.I., Elston, R.C.  Wilson, A.F.  and Bailey-Wilson, J.E.  (1989).
A more powerful robust sib-pair test of linkage for quantitative traits.  GE
6:435-49.

\item Amos I.C.  (1994).  Robust variance-components approach for assessing
genetic linkage in pedigrees, AJHG 54:  515-41.

\item Goldgar, D.E.  (1990).  Multipoint analysis of human quantitative genetic
variation.  AJHG 47:957-67.

\item Li, C.C. (1955). Population Genetics, University of Chicago Press,
Chicago.

\item Olsen, J.M.  (1995).  Robust multipoint linkage analysis:  an extension
of Haseman-Elston method, GE 12:177-93.

\item Tran, L.D., Elston, R.C., Keats, B.J.B.  and Wilson, A.F.  (1994).
Sib-pair linkage program (SIBPAL) User's Guide.

\item Terwilliger, J.D.  and Ott, J.  (1994).  Handbook of Human Genetic
Linkage, The Johns Hopkins University Press, Baltimore and London.

\item Young, A. (1995). GAS Manual, Oxford University.

\item Hodge, S.E.  (1984).  The information contained in multiple sibling
pairs, GE 1:109-22.

\item Risch, N.  and Zhang, H.(1995), Extreme discordant sib pairs for mapping
quantitative trait loci in humans, Science, vol 268:1584-9.

\item Lange, K.  (1986a).  The affected sib pair method using identity by state
relations.  AJHG 50:  148-150.

\item Lange, K.  (1986b).  A test statistic for the affected-sib- set method,
AHG 50:283-90.

\item Suarez, B.K., Rice, J.  and Reich, T.(1978).  The generalized sib pair IBD
distribution:  its use in the detection of linkage, AHG 42:87-94.

\item Boehnke, M.  (1990).  Sample-size guidelines for linkage analysis of a
dominant locus for a quantitative trait by the method of lod scores, AJHG
47:218-27.

\end{enumerate}
\end{document}
